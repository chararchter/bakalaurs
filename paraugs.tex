% Compile with: latexmk -pdfxe -outdir=build
\documentclass[12pt,oneside]{LuThesis}

\title{Saules paneļu efektivitāte Latvijas klimatā}
\author{Viktorija Leimane}

%%% LuThesis deklerācijas
\def\studaplnum{vl16047}
\def\darbvad{Dr. Phys. Andris Jakovičs}
\def\dokfak{\textbf{Fizikas, matemātikas un optometrijas fakultāte}}
\def\recenzents{Dr. Phys.Aivars Vembris}
\def\doktitle{\textbf{Saules paneļu efektivitāte Latvijas klimatā}}
\def\dokfak{\textbf{Fizikas, matemātikas un optometrijas fakultāte}}
\def\dokdate{} % Darba vadītāja parakstīšanas datums
\def\dokdateiesn{} % Darba iesniegšanas datums

%%% Valodas un fonti
\setdefaultlanguage{latvian}
\setotherlanguage{english}
\setmainfont[BoldFont=FreeSerifBold.ttf,ItalicFont=FreeSerifItalic.ttf,BoldItalicFont=FreeSerifBoldItalic.ttf]{FreeSerif.ttf}

\usepackage{graphicx}

%%% Misc
% \usepackage{hyperref}
\usepackage[hidelinks]{hyperref}
\addbibresource{bachelor.bib} % biblatex bibliotēkas fails: bachelor.bib
\usepackage[section]{placeins}

%%% Dokumenta struktūra
\begin{document}

\maketitle

% * Anotācija
\chapter*{Anotācija}
\setcounter{page}{1}
\begin{abstract}
Darba mērķis ir noteikt efektīvāko saules paneļu izvietojuma veidu Latvijai tipiskos meteoroloģiskajos apstākļos. 
Balstoties uz divu veidu saules paneļiem, kas novietoti piecās dažādās telpiskajās orientācijās Latvijas Universitātes Botāniskā dārza teritorijā, tiks noteikta solāro paneļu efektivitātes atkarība no mainīgiem parametriem:
1) meteoroloģiskie apstākļi
2) telpiskā orientācija
3) gada mēnesis
4) solāro paneļu tips. \\
Iegūtie monitoringa rezultāti tiks analizēti kontekstā ar šo paneļu efektivitātes fizikālo novērtējumu.\\

\keywords{Saules enerģijas paneļi, atjaunojamo energoresursu enerģija, vides monitorings}
\end{abstract}

\chapter*{Abstract}
\begin{english}
\begin{abstract}
The aim of this thesis is to determine the most efficient way of solar panel arrangement for the climatic conditions of Latvia.
Based on two types of solar panels placed in five different spatial orientations in the University of Latvia Botanical Garden area, the dependency of the efficiency of solar panels on following variable parameters is established:
\begin{enumerate}
\item type of solar panels
\item spatial orientation
\item month of year
\item meteorological conditions.
\end{enumerate}

The results of the monitoring are analysed in the context of solar irradiance intensity, the physical assessment of the partial(?) efficiency of the panels and the results of other measurements.

\keywords{Solar panels, renewable energy}
\end{abstract}
\end{english}

%* Saturs
\tableofcontents

%* Apzīmējumu saraksts
\chapter*{Apzīmējumu saraksts}
\addcontentsline{toc}{chapter}{Apzīmējumu saraksts}
\noindent 
\textbf{Klimats}\\
TSI - Kopējais saules apstarojums, $\textrm{Wm}^{-2}$\\
SSI - Saules spektrālais apstarojums, $\textrm{Wm}^{-2}\textrm{nm}^{-1}$\\
% $G_{sc}$ - Solārā konstante, $\textrm{Wm}^{-2}$\\
%Photovoltaic power potential
TIM - Kopējā apstarojuma novērotājs\\
GHI – Globālais horizontālais apstarojums,  $\textrm{kWh/m}^2$\\ %Global horizontal irradiation
DHI – Difūzais horizontālais apstarojums,  $\textrm{kWh/m}^2$\\ 
DNI – Tiešais normālais apstarojums, $\textrm{kWh/m}^2$\\ %Direct normal irradiation
CMF - Mākoņu modifikācijas reizinātājs\\
AU - astronomiskā vienība\\  %astronomical unit
\textbf{Saules kustības leņķi}\\
$\theta$ - staru krišanas leņķis uz saules paneli\\
$\delta$ - Saules deklinācija -- leņķis starp virzieniem uz Sauli un debess ekvatoru solārajā pusdienlaikā.\\
 $\phi$  - ģeogrāfiskais platums; pozitīvs Z virzienā.\\
$\beta$  - paneļa slīpums -- leņķis starp Saules paneļa virsmu un horizontāli.\\
$\gamma$ - paneļa azimuts -- leņķis starp virsmas normāles projekciju uz horizontālo  plakni un D virzienu.\\
$\omega$ - solārais stundu leņķis -- leņķis starp Saules stara virziena projekciju uz horizontālo plakni un D virzienu; negatīvs no rīta.\\
\textbf{Saules paneļi}\\
PV - fotoelektriskais elements\\ %fotogalvanisks?
Si - silīcijs\\
P - jauda, 	W\\
Pmax - maksimālā jauda, W\\
PVOUT – Saules fotoelementa potenciālā jauda, $\textrm{kWh/kWp}$\\ 
%Photovoltaic power potential
E$_{norm}$, $\textrm{kWh/m}^2$ - enerģija normēta uz saules paneļa laukuma vienību.\\
\textbf{Debespuses}\\
A - austrumi\\
R - rietumi\\
D - dienvidi\\

%* Ievads
\chapter*{Ievads}
\addcontentsline{toc}{chapter}{Ievads}
Apvienoto Nāciju Organizācijas Klimata konferencē Parīzē 2015. gada decembrī daudzas pasaules valstis vienojās ierobežot globālo sasilšanu zem 2\textdegree C salīdzinājumā ar pirmsindustriālo laikmetu.
%, tāpēc ES ir apņēmusies līdz 2030. gadam samazināt siltumnīcefekta gāzu emisijas vismaz par 40\% salīdzinājumā ar 1990. gada līmeni.
Tāpēc Eiropas Savienībā noteikts dalībvalstīm saistošs mērķrādītājs  –  vismaz 32\%  atjaunojamās enerģijas īpatsvars līdz 2030. gadam \cite{ES}.

Ne mazāk būtiska ir atjaunojamo energoresursu loma energoapgādes neatkarības un drošības veicināšanā, tehnoloģiju attīstībā un inovācijās, vienlaikus sniedzot labumu videi un sabiedrībai, kā arī nodrošinot svarīgus priekšnosacījums nodarbinātībai, reģionālajai attīstībai un elektrības nodrošināšanai grūti pieejamās vietās \cite{ES}.

Dažādu atjaunojamo enerģijas resursu - saules, vēja, ģeotermālo un ūdens - starpā Saules enerģija ir viens no kandidātiem klimata pārmaiņu un to seku mazināšanai un efektīvas energoapgādes nodrošināšanai. Pēdējā desmitgadē veiktās investīcijas Saules enerģijas izmantošanā manifestējās inovācijās saules paneļu ražošanā, un gala rezultātā tie ir kļuvuši efektīvāki un finansiāli pieejamāki patērētājiem, piemēram, silīcija saules paneļu cena sastāda $\leq$ 30\% no kopējām saules paneļu sistēmas uzstādīšanas izmaksām un to saražotā enerģija atkarībā no atrašanās vietas un paneļa veida atmaksājas pēc $\geq$ 3 gadu perioda \cite{researchOpp}. Tomēr bez klimata to efektivitāti ietekmē arī daudzi citi faktori, kas tiek analizēti šajā darbā, piemēram, saules paneļu tips un telpiskā orientācija.

Šī pētījuma \textbf{mērķis} ir analizēt un praktiski pārbaudīt divu tipu (JA un LG) saules paneļu efektivitāti atšķirīgos telpiskās orientācijās risinājumos -- pētītas trīs dažādu virzienu (dienvidu (D), rietumu (R), austrumu (A)) un trīs leņķu pret horizontu (13\textdegree, 40\textdegree, 90\textdegree) paneļu grupas -- un tiek salīdzināta to piemērotība Latvijas klimatiskajiem apstākļiem.

% \subsection{Darba uzdevumi}
\textbf{Darba uzdevumi}
\begin{itemize}
\item Ievākt, atlasīt un analizēt saules paneļu jaudas (P) datus.
\item Izveidot iespējami automatizētu datu apstrādes sistēmu R valodā ilgtermiņa montioringa vajadzībām.
\item Salīdzināt paneļu efektivitāti gada laikā mēnešu intervālos atbilstoši tipa un telpiskās orientācijas apakšgrupām.
% \item Balstoties uz ilgtermiņa saules izstarojuma monitoringu, novērtēt saules paneļu saražoto enerģiju no fizikālajiem apsvērumiem.
\item Novērtēt datu kvalitāti no fizikālo apsvērumu un saules apstarojuma mērījumu viedokļa.
\end{itemize}

% \subsection{Darba aktualitāte}
\textbf{Darba aktualitāte}

Darba nozīme atklājas detalizētas paneļu efektivitātes analīzes nepieciešamībā tieši Latvijas klimatiskajiem apstākļiem. Latvijā veikto pētījumu daudzums un kvalitāte pagaidām neļauj iegūt pilnīgu priekšstatu par Saules paneļu lietošanas iespējām un prognozēt dažādu paneļu tipu efektivitāti reālā Latvijas klimatā. Tāpēc šī darba novitāte ietverta programmatūras izveidē, kas ļauj attēlot un apstrādāt Saules paneļu monitoringa datus, kas dos iespēju veikt turpmākus dziļākus pētījumus par dažādiem saules paneļu spēkstaciju pielietojuma aspektiem Latvijā.\\
% \subsection{Darba struktūra}

\textbf{Darba struktūra}

Darba pirmo daļu veido literatūrā pieejamo Saules apstarojuma novērtējumu raksturojums un apskats par Saules redzamo pārvietošanās pie debess sfēras diennakts laikā 60 dienu intervālos. Otrajā daļā aplūkota saules paneļu uzbūve un to darbības princips. Trešajā daļā tiek apskatīta saules paneļu sistēmas elektriskā shēma un raksturoti LU Botāniskā dārza spēkstacijas instalācijas parametri.
Ceturtajā daļā ir aprakstīti iegūtie rezultāti, tie ir salīdzināti savā starpā, ar citas saules paneļu spēkstacijas mērījumu rezultātiem, ar eksperimentālā poligona meteostacijas datiem par solāro apstarojumu šajā laika periodā, kā arī aplūkoti no teorētiskās pusvadītāju fizikas aspektiem.\\

\textbf{Autores ieguldījums darbā}
% \subsection{Autores ieguldījums darbā}
% the \\ insures the section title is centered below the phrase: Appendix B

Saules paneļu uzstādīšanu veica SIA EG Inženieri Valda Gailīša vadībā.
Par datu ievākšanu no saules paneļu sistēmas datu uzkrājējiem jāpateicas Victron Energy B.V. izstrādātājai VRM sistēmai.

Mans ieguldījums ir koda arhitektūras plānošana, rakstīšana un uzturēšana līdz šim brīdim. Esmu vienīgais koda bāzes izstrādātājs. Darbā kopā veikti 318 iesūtījumi (\textit{commits}) trīs zaros (105, 123 un 90 iesūtījumi attiecīgi), kas rezultējās 3140 koda rindās. 

Darba gaitā veicu datu lejupielādi, datu informācijas aptveršanu, kā arī datu programmatisku atlasīšanu, apstrādi, apkopošanu, transformēšanu vizualizācijas vajadzību pielāgošanai un datu vizualizāciju. 

% \begin{figure}[h]
% 	\centering
% 	\includegraphics[width=0.6\linewidth]{figures/misc/contributions.png}
% 	\caption{Darba autores ieguldījums master zarā.}
% 	\label{fig:retribution}
% \end{figure}
% TEORIJAS KOPSAVILKUMS
Vispirms tiek aplūkota Saules emitētā starojuma daba un ģeometriskie apsvērumi - virziens, no kura staru kūlis sasniedz virsmu, leņķis uz virsmas un laika gaitā saņemtais starojuma daudzums. Tiek apskatīta atmosfēras un mākoņu ietekme uz virsmas saņemto saules starojumu, un tās praktiskā nozīme, apstrādājot pieejamos Saules starojuma datus, lai aprakstītu radiācijas gadījumus uz virsmas dažādās orientācijās.

% atjaunojamā enerģija
% kāpēc ir svarīgi
% 	politika (direktīva)
% 	klimata pārmaiņas
% atjaunojamās enerģijas veidi (vējš, saule, utml)
% mērķis ir salīdzināt solāro paneļu saražoto enerģiju atkarībā no modeļa, telpiskās orientācijas, leņķa un kā mainās pa gadalaikiem tieši latvijā
% citos klimatiskajos apstākļos līdzīgas analīzes ir veiktas ko var iegūt dažādās klimatiskajās zonās
% pierakstīt par pv paneļu darbību un efektivitāti
% aprakstīt ko tu darīji, lai pēc iespējas automatizētu datu apstrādes sistēmu
% kādus datus uzkrāj kādā attēlojumā
% datu menedžmentu aprakstīt
% paskatīties kā nosaka solāro paneļu efektivitātes standartu
% sarēķināt attiecības starp paneļu saražoto (normēt uz lielāko paneli)
% ielikt grafikus no februāra un aprīļa
% pierakstīt salīdzinājumu
% aprakstīt gļukus, kur dienā nav nekā saražots bet blakus ir. tā gadās.
% nomaini februāra sol_month grafikos virsrakstu uz februāri
% nomaini $Em^-2$ uz $E_norm$

% * Literatūras apskats
\chapter{Literatūras apskats}
\input{tex/literatura}
% intensity
% spectral distribution
% solar geometry
% saules stāvoklis debesīs
% un virziens, kurā stara starojums krīt uz dažādu virzienu un ēnojuma virsmām

\section{Saules apstarojums}

Lielākā daļa Saules emitētās enerģijas tiek saražota kodolreakcijās fotosfērā.

% Saņemto enerģiju laika vienībā uz uz laukuma vienības perpendikulāri starojuma izplatīšanās virzienam 1 AU attālumā integrēta pa visiem viļņu garumiem raksturo solārā konstante ($G_{sc}$).

Solārā konstante $G_{sc}$ ir saņemtā enerģija laika vienībā uz laukuma vienības perpendikulāri starojuma izplatīšanās virzienam 1 AU attālumā integrēta pa visiem viļņu garumiem.\cite{ThermalProcesses}

Kopējā saules apstarojuma vērtība mainās laikā un korelē ar Saules plankumu ciklu.
Viena fizikālā lieluma tikai daļēji pārklājušos novērojumu laikrindu apvienošana kompozītā ir gan zinātnisks, gan statistisks izaicinājums un neviens kompozīts (piemēram, PMOD, ACRIM, IRBM) līdz šim nav guvis konsensu solārā apstarojuma pētnieku kopienā.~\ref{fig:TSI_misijas}

Par labāko saules apstarojuma mērījumu reprezentāciju tiek uzskatīti TIM instrumenta dati mēraparāta uzbūves (TIM ir lielāka precizitātes apertūra tuvu dobumam un mazāka redzeslauku bloķējošā pie instrumenta ieejas, klasiskajos radiometros ir pretēji, kas palielina jutību pret atstaroto gaismu no redzeslauku bloķējošās apertūras un kļūdu no instrumenta sasildīšanas) un augstās precizitātes dēļ, tāpēc šajā darbā grafiki balstās uz šiem mērījumiem, pēc kuriem absolūtā kopējā saules apstarojuma vērtība ir $1360.8 \pm 0.5 \textrm{Wm}^{-2}$.\cite{Frohlich2012}

\begin{figure}[h]
    \centering
    \includegraphics[width=0.6\linewidth]{figures/misc/TSI_misijas.png}
    \caption{Salīdzinājums dienā vidējotiem saules kopējā apstarojuma datiem no dažādām misijām un Saules plankuma skaitlis, lai ilustrētu solārās aktivitātes variabilitāti trīs ciklos. \cite{Frohlich2012}}
    \label{fig:TSI_misijas}
\end{figure}

\begin{table}[h]
    \caption{TSI mērījumu vēsture} % caption iet pirms tabulas
    \begin{center}
    \begin{tabular}{| r | c | l |}
    \hline
    radiometrs & misija & darbības laiks \\ \hline
    Hickey-Frieden & NIMBUS-7 & 1978--1992  \\ \hline
	ACRIM I & Solārā Maksimuma Misija (SMM) & 1980--1989 \\ \hline
	ACRIM  & Zemes Radiācijas Budžeta Satelīts (ERBS) & 1984--2003 \\ \hline
	ACRIM II & Augšējās Atmosfēras Izpētes Satelīts (UARS) & 1991--2001 \\ \hline
	VIRGO & Solārā un Heliosfēras observatorija (SOHO)& 1996--pašlaik \\ \hline
	ACRIM III & ACRIMSAT  & 2000--pašlaik \\ \hline
	TIM & Saules Radiācijas un Klimata Eksperiments (SORCE) & 2003--pašlaik\\ \hline
    \end{tabular}
    \end{center}
    \label{tab:radiometers}
\end{table}


% Kopējais saules apstarojums (TSI) ir saules starojuma absolūtās intensitātes mērījums integrēts visā saules enerģijas diskā un visā saules enerģijas spektrā.
% diennakts vidējais apstarojums 1 AU attālumā no Saules.
% Norāda uz solārās radiācijas izmaiņām, kas ietekmē solārās enerģijas apjomu uz Zemes atmosfēras augšējiem slāņiem.


\begin{figure}[h]
    \centering
    \includegraphics[width=\linewidth]{figures/misc/TSI_8-19.pdf}
    \caption{Kopējais saules apstarojums 24. saules ciklā \cite{TSIdata}}
    \label{fig:TSI1}
\end{figure}

\begin{figure}[h]
    \centering
    \includegraphics[width=\linewidth]{figures/misc/TSI.pdf}
    \caption{Kopējais saules apstarojums solāro paneļu datu ieguves laikā \cite{TSIdata}}
    \label{fig:TSI2}
\end{figure}

\begin{figure}[h]
    \centering
    \includegraphics[width=\linewidth]{figures/misc/LV_DNI.png}
    \caption{Tiešais normālais apstarojums \cite{solargis}}
    \label{fig:lv_DNI}
\end{figure}
\begin{figure}[h]
    \centering
    \includegraphics[width=\linewidth]{figures/misc/LV_GHI.png}
    \caption{Globālais horizontālais apstarojums Latvijā \cite{solargis}}
    \label{fig:lv_GHI}
\end{figure}
\begin{figure}[h]
    \centering
    \includegraphics[width=\linewidth]{figures/misc/LV_PVOUT.png}
    \caption{PV potenciālā jauda \cite{solargis}}
    \label{fig:lv_PVOUT}
\end{figure}


%* Rezultāti
\chapter{Rezultāti un diskusija}
\input{tex/rezultati}

%* Secinājumi
\chapter*{Secinājumi}
\addcontentsline{toc}{chapter}{Secinājumi}
Darba laikā tika apgūtas ggplot2, tidyr, lubridate un dplyr bibliotēkas, ar kuru palīdzību tika atlasīts, analizēts un apkopots liels datu apjoms par 10 saules paneļu darbību no 2019. gada 1. janvāra līdz 30. aprīlim.

Darbā iegūtie rezultāti ļauj izdarīt secinājumus, ka no sistēmā esošajiem parametriem efektīvākā kombinācija ir:
\begin{itemize}
	\item 40 grādu leņķis
	\item D virziens
	\item LG panelis
	\item aprīļa mēnesis
\end{itemize}

% KĻŪDAS!!!?????????

Saules paneļu efektivitātes dziļāka izpratne pieprasa tālākus pētījumus, it īpaši nolietojuma, putekļu, vēja, nokrišņu un citu apstākļu ietekmes izpētei.

%* Pateicības
\chapter*{Pateicības}
\addcontentsline{toc}{chapter}{Pateicības}
Pateicos paroksetīnam, xanax, GNU/Linux, Pētera Draguna dzejas krājumam 'Tumšās stundas', Tarvi Verro for teaching me git, Valtam Krūmiņam par emocionālo atbalstu, Žeņam par kucēnu video, Solvitai par maģiju un Cilvēkam par pacietību. Paldies "Puratos Latvia" un Asjas un Berndta Everts piemiņas fondam par stipendiju studiju laikā.

Darbs veikts ar Eiropas Reģionālās attīstības fonta projekta "Viedo risinājumu gandrīz nulles enerģijas ēkām izstrāde, optimizācija un ilgtspējas izpēte reāla klimata apstākļos" Nr ESS2017/209 1.1.1.1/16/A/192 finansiālo atbalstu.


%* Izmantotā darba literatūra un avoti
\clearpage
\addcontentsline{toc}{chapter}{Izmantotā darba literatūra un avoti}
\printbibliography[title=Izmantotā darba literatūra un avoti]

%%% Pielikumi
\appendix

% Dokumentālā lapa
\makedoklapa

\end{document}
