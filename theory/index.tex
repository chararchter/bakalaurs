% TEORIJAS KOPSAVILKUMS
Vispirms tiek aplūkota Saules emitētā starojuma daba un ģeometriskie apsvērumi - virziens, no kura staru kūlis sasniedz virsmu, leņķis uz virsmas un laika gaitā saņemtais starojuma daudzums. Tiek apskatīta atmosfēras ietekme uz virsmas saņemto saules starojumu, un tās praktiskā nozīme, apstrādājot pieejamos Saules starojuma datus, lai aprakstītu radiācijas gadījumus uz virsmas dažādās orientācijās.