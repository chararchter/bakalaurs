Apvienoto Nāciju Organizācijas Klimata konferencē Parīzē 2015. gada decembrī puses no visas pasaules vienojās ierobežot globālo sasilšanu zem 2 ºC salīdzinājumā ar pirmsindustriālo laikmetu.
%, tāpēc ES ir apņēmusies līdz 2030. gadam samazināt siltumnīcefekta gāzu emisijas vismaz par 40\% salīdzinājumā ar 1990. gada līmeni.
Tāpēc Eiropas Savienībā noteikts dalībvalstīm saistošs mērķrādītājs  –  vismaz 32\%  atjaunojamās enerģijas īpatsvars līdz 2030. gadam.\cite{ES}

Ne mazāk būtisks ir atjaunojamo energoresursu pienesums energoapgādes neatkarības un drošības veicināšanā, tehnoloģiju attīstībā un inovācijās, vienlaikus sniedzot labumu videi un sabiedrībai, kā arī nodrošinot svarīgus priekšnosacījums nodarbinātībai, reģionālajai attīstībai un elektrības nodrošināšanai grūti pieejamās vietās. \cite{ES}

Saules enerģija ir labs kandidāts mazināt klimata pārmaiņu sekas un nodrošināt efektīvu energoapgādi. Pēdējā desmitgadē veiktās investīcijām Saules enerģijā manifestējās inovācijās saules paneļu ražošanā, un gala rezultātā tie ir kļuvuši finansiāli pieejamāki patērētājiem, piemēram, uz silīcija balstītu paneļu cena sastāda mazāk nekā 30\% no kopējām saules paneļu sistēmas uzstādīšanas izmaksām un to saražotā enerģija atmaksājas vidēji trīs gadu periodā.\cite{researchOpp}

Šī pētījuma mērķis ir analizēt un praktiski pārbaudīt divu tipu (JA un LG) saules paneļu atšķirīgos uzstādījuma risinājumus -  trīs virzienu (dienvidi, rietumi, austrumi) un trīs leņķu (13, 40, 90) grupas - un to piemērotību Latvijai tipiskiem meteoroloģiskajiem apstākļiem.

\textbf{Darba uzdevumi}
\begin{itemize}
\item Ievākt, atlasīt un analizēt saules paneļu datus un novērtēt to kvalitāti.
\item Izveidot iespējami automatizētu datu apstrādes sistēmu R valodā.
\item Salīdzināt paneļu efektivitāti atbilstoši to parametru apakšklasēm.
\end{itemize}

\textbf{Darba struktūra}\\
Darba pirmā daļā sastāv no literatūrā pieejamo Saules apstarojuma novērtējumu pamatojuma un teorijas apskata par Saules redzamo pārvietošanās pie debess sfēras diennakts laikā dažādos mēnešos. Otrajā daļā aplūkota saules paneļu uzbūve un darbības princips, kā arī sistēmas shēma un uzstādījumu parametru raksturojums. Trešā daļā ir apskatīti rezultāti aprakstītajiem saules paneļiem, tie salīdzināti savā starpā un ar eksperimentālā poligona meteostacijas stacijas datiem par solāro apstarojumu šajā laika periodā.


% un lai to analīzi veiktu ir darīti uzdevumi
% uzstādīti paneļi (profesionāļi to darīja)
% Ievākti un analizēti dati
% Datu kvalitātes novērtējums
% novērtēt vai iegūtie resultāti ir ticami

% citos klimatiskajos apstākļos līdzīgas analīzes ir veiktas ko var iegūt dažādās klimatiskajās zonās 
% pierakstīt par pv paneļu darbību un efektivitāti
% aprakstīt ko tu darīji, lai pēc iespējas automatizētu datu apstrādes sistēmu
% kādus datus uzkrāj kādā attēlojumā
% datu menedžmentu aprakstīt