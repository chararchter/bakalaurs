Apvienoto Nāciju Organizācijas Klimata konferencē Parīzē 2015. gada decembrī daudzas pasaules valstis vienojās ierobežot globālo sasilšanu zem 2 ºC salīdzinājumā ar pirmsindustriālo laikmetu.
%, tāpēc ES ir apņēmusies līdz 2030. gadam samazināt siltumnīcefekta gāzu emisijas vismaz par 40\% salīdzinājumā ar 1990. gada līmeni.
Tāpēc Eiropas Savienībā noteikts dalībvalstīm saistošs mērķrādītājs  –  vismaz 32\%  atjaunojamās enerģijas īpatsvars līdz 2030. gadam \cite{ES}.

Ne mazāk būtiska ir atjaunojamo energoresursu loma energoapgādes neatkarības un drošības veicināšanā, tehnoloģiju attīstībā un inovācijās, vienlaikus sniedzot labumu videi un sabiedrībai, kā arī nodrošinot svarīgus priekšnosacījums nodarbinātībai, reģionālajai attīstībai un elektrības nodrošināšanai grūti pieejamās vietās \cite{ES}.

Dažādu pieejamo atjaunojamo resursu - Saule, vējš, zeme, ūdens - starpā Saules enerģija ir viens no kandidātiem klimata pārmaiņu un to seku mazināšanai un efektīvas energoapgādes nodrošināšanai. Pēdējā desmitgadē veiktās investīcijas Saules enerģijas izmantošanā manifestējās inovācijās saules paneļu ražošanā, un gala rezultātā tie ir kļuvuši efektīvāki un finansiāli pieejamāki patērētājiem, piemēram, silīcija saules paneļu cena sastāda mazāk nekā 30\% no kopējām saules paneļu sistēmas uzstādīšanas izmaksām un to saražotā enerģija atmaksājas vidēji trīs gadu periodā \cite{researchOpp}.

Tomēr bez klimata to efektivitāti ietekmē arī daudzi citi faktori, to skaitā telpiskā orientācija.

Šī pētījuma \textbf{mērķis} ir analizēt un praktiski pārbaudīt divu tipu (JA un LG) saules paneļu efektivitāti atšķirīgos telpiskās orientācijās risinājumos -- pētītas trīs dažādu virzienu (dienvidi, rietumi, austrumi) un trīs leņķu pret horizontu (13\textdegree, 40\textdegree, 90\textdegree) paneļu grupas -- un tiek salīdzināta to piemērotība Latvijas klimatiskajiem apstākļiem.

\textbf{Darba uzdevumi}
\begin{itemize}
\item Ievākt, atlasīt un analizēt saules paneļu jaudas (P) datus.
\item Izveidot iespējami automatizētu datu apstrādes sistēmu R valodā ilgtermiņa montioringa vajadzībām.
\item Salīdzināt paneļu efektivitāti gada laikā mēnešu intervālos atbilstoši tipa un telpiskās orientācijas apakšgrupām.
% \item Balstoties uz ilgtermiņa saules izstarojuma monitoringu, novērtēt saules paneļu saražoto enerģiju no fizikālajiem apsvērumiem.
\item Novērtēt datu kvalitāti no fizikālo apsvērumu un saules apstarojuma mērījumu viedokļa.
\end{itemize}

\textbf{Darba struktūra}\\
Darba pirmo daļu veido literatūrā pieejamo Saules apstarojuma novērtējumu raksturojums un apskata par Saules redzamo pārvietošanās pie debess sfēras diennakts laikā janvāra un aprīļa mēnešos. Otrajā daļā aplūkota saules paneļu uzbūve un darbības princips, kā arī apskatīta sistēmas shēma un saues paneļu konkrētās instalācijas LU Botāniskajā dārzā parametru raksturojums.
Trešā daļā ir aprakstīti iegūtie rezultāti, tie ir salīdzināti savā starpā, ar cita saules paneļa uzstādījuma mērījumu rezultātiem un ar eksperimentālā poligona meteostacijas datiem par solāro apstarojumu šajā laika periodā.


% un lai to analīzi veiktu ir darīti uzdevumi
% uzstādīti paneļi (profesionāļi to darīja)
% Ievākti un analizēti dati
% Datu kvalitātes novērtējums
% novērtēt vai iegūtie resultāti ir ticami

% citos klimatiskajos apstākļos līdzīgas analīzes ir veiktas ko var iegūt dažādās klimatiskajās zonās 
% pierakstīt par pv paneļu darbību un efektivitāti
% aprakstīt ko tu darīji, lai pēc iespējas automatizētu datu apstrādes sistēmu
% kādus datus uzkrāj kādā attēlojumā
% datu menedžmentu aprakstīt