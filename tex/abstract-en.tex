The aim of this thesis is to determine the most efficient way of solar panel arrangement for the climatic conditions of Latvia.
Based on two types of solar panels placed in five different spatial orientations in the University of Latvia Botanical Garden area, the dependency of the efficiency of solar panels on following variable parameters is established:
\begin{enumerate}
\item type of solar panels
\item spatial orientation
\item month of year
\item meteorological conditions.
\end{enumerate}

The results of the monitoring are analysed in the context of solar irradiance intensity, the physical assessment of the partial(?) efficiency of the panels and the results of other measurements.

\keywords{Solar panels, renewable energy}