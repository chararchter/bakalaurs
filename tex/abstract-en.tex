The aim of this thesis is to determine optimal solar panel arrangement for the climatic conditions of Latvia.
By studying two types of solar panels placed in five different spatial orientations in the University of Latvia Botanical Garden area, the dependency of solar panel efficiency on following variable parameters is established:
\begin{enumerate}
\item Type of solar panels (JA or LG)
\item Spatial orientation (W.13, E.13, S.13, S.40, S.90)
\item Month of year (january - april)
\item Meteorological conditions (solar irradiance)
\end{enumerate}

The results of the monitoring are analysed in the context of solar irradiance intensity, the physical assessment of the potential efficiency of the panels and the results of other measurements.
The results show that the optimal parameters are LG panel at the 40 degree angle of south direction. The study data analysis tool is available in \url{https://github.com/chararchter/solR}.

\keywords{Solar panels, renewable energy}