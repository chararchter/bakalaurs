\section{Definīcijas}
\noindent \textbf{Saules plankums} - magnētiskās plūsmas koncentrācija bipolāros klāsteros vai grupās, kas novērojama kā tumšs plankums uz Saules fotosfēras.\\
\textbf{Saules plankumu cikls} - aptuveni 11 gadus ilga kvaziperiodiska variācija saules plankuma skaitlī. Magnētiskā lauka polaritātes modelis mainās ar katru ciklu.\\
\textbf{Saules plankuma skaitlis} - Dienas saules plankuma aktivitātes indekss (R), definēts kā $R = k(10 \cdot g + s)$, kur
s - individuālo plankumu skaits;
g - saules plankumu grupu skaits;
k - observatorijas faktors.\\
\textbf{SSI} - saules spektrālais starojums vai spektra enerģijas blīvums - saules jaudas izkliede uz virsmas laukuma vienību.\\
% total solar irradiance (TSI) - Solar energy per unit time over a unit area perpendicular to the Sun’s rays at the top of Earth’s atmosphere.
\textbf{TSI} - Saules starojuma absolūtās intensitātes mērījums integrēts visā saules enerģijas diskā un visā saules enerģijas spektrā.\\
% Laboratory for Atmospheric and Space Physics, University of Colorado (2019)
% \\http://lasp.colorado.edu/home/sorce/reference/glossary/
\textbf{Izstarojums} - starojuma avota jaudas incidents uz virsmas laukuma vienību.\\ %Irradiance
\textbf{Saules diennakts kustība} (diurnal motion) - Debess spīdekļu redzamā pārvietošanās pie debess sfēras (rotācija ap pasaules asi) diennakts laikā.\\ %Diurnal motion
% Tās faktiskais cēļonis ir Zemes rotācija ap asi. Diennakts kurstībā visi debess spīdekļi pārvietojas pa debess paralēlēm.
Beam Radiation - the solar radiation received from the sun without having been scattered by the atmosphere (also known as direct solar radiation)
Diffuse Radiation - the solar radiation received from the sun after its direction has been changed by scattering by the atmosphere