\section{Definīcijas}
\noindent \textbf{Saules plankums} - magnētiskās plūsmas koncentrācija bipolāros klāsteros vai grupās, kas novērojama kā tumšs plankums uz Saules fotosfēras.\\
\textbf{Saules plankumu cikls} - aptuveni 11 gadus ilga kvaziperiodiska variācija saules plankuma skaitlī. Magnētiskā lauka polaritātes modelis mainās ar katru ciklu.\\
\textbf{Saules plankuma skaitlis} - Dienas saules plankuma aktivitātes indekss (R), definēts kā R = k(10g + s), kur
s - individuālo plankumu skaits;
g - saules plankumu grupu skaits;
k - observatorijas faktors.\\
\textbf{SSI} - saules spektrālais starojums vai spektra enerģijas blīvums - saules jaudas izkliede uz virsmas laukuma vienību.\\
% total solar irradiance (TSI) - Solar energy per unit time over a unit area perpendicular to the Sun’s rays at the top of Earth’s atmosphere.
\textbf{TSI} - Saules starojuma absolūtās intensitātes mērījums integrēts visā saules enerģijas diskā un visā saules enerģijas spektrā.
% \\http://lasp.colorado.edu/home/sorce/reference/glossary/