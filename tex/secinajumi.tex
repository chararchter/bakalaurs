Darba laikā tika apgūtas ggplot2, tidyr, lubridate un dplyr bibliotēkas, ar kuru palīdzību tika atlasīts, analizēts un apkopots liels datu apjoms par 10 saules paneļu darbību no 2019. gada 1. janvāra līdz 30. aprīlim.

Darbā iegūtie rezultāti ļauj izdarīt secinājumus, ka no sistēmā esošajiem parametriem efektīvākā kombinācija ir:
\begin{itemize}
	\item 40 grādu leņķis
	\item D virziens
	\item LG panelis
	\item aprīļa mēnesis
\end{itemize}

% KĻŪDAS!!!?????????

Saules paneļu efektivitātes dziļāka izpratne pieprasa tālākus pētījumus, it īpaši nolietojuma, putekļu, vēja, nokrišņu un citu apstākļu ietekmes izpētei.