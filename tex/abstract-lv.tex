Darba mērķis ir noteikt efektīvāko saules paneļu izvietojuma veidu Latvijas klimatiskajos apstākļos. 
Balstoties uz divu veidu saules paneļiem, kas novietoti piecās dažādās telpiskajās orientācijās Latvijas Universitātes Botāniskā dārza teritorijā, tiek noteikta solāro paneļu efektivitātes atkarība no mainīgiem parametriem:
\begin{enumerate}
\item solāro paneļu tips
\item telpiskā orientācija
\item gada mēnesis
\item meteoroloģiskie apstākļi.
\end{enumerate}

Iegūtie monitoringa rezultāti tiek analizēti kontekstā ar saules izstarojuma intensitāti, paneļu potenciālās efektivitātes fizikālo novērtējumu un citu mērījumu rezultātiem.\\

\keywords{Saules enerģijas paneļi, atjaunojamo energoresursu enerģija}