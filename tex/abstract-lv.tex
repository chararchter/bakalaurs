Darba mērķis ir noteikt efektīvāko saules paneļu izvietojuma veidu Latvijai tipiskos meteoroloģiskajos apstākļos. 
Balstoties uz divu veidu saules paneļiem, kas novietoti piecās dažādās telpiskajās orientācijās Latvijas Universitātes Botāniskā dārza teritorijā, tiks noteikta solāro paneļu efektivitātes atkarība no mainīgiem parametriem:
1) meteoroloģiskie apstākļi
2) telpiskā orientācija
3) gada mēnesis
4) solāro paneļu tips. \\
Iegūtie monitoringa rezultāti tiks analizēti kontekstā ar šo paneļu efektivitātes fizikālo novērtējumu.\\

\keywords{Saules enerģijas paneļi, atjaunojamo energoresursu enerģija, vides monitorings}