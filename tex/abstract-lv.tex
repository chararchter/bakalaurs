Darba mērķis ir noteikt optimālo saules paneļu izvietojuma veidu Latvijas klimatiskajos apstākļos. 
Pētot divu veidu saules paneļus, kas novietoti piecās dažādās telpiskajās orientācijās Latvijas Universitātes Botāniskā dārza teritorijā, tiek noteikta solāro paneļu efektivitātes atkarība no parametriem:
\begin{enumerate}
\item Saules paneļu tips (JA vai LG)
\item Telpiskā orientācija (R.13, A.13, D.13, D.40, D.90)
\item Gada mēnesis (janvāris - aprīlis)
\item Meteoroloģiskie apstākļi (saules apstarojums).
\end{enumerate}

Iegūtie monitoringa rezultāti tiek analizēti kontekstā ar saules izstarojuma intensitāti, paneļu potenciālās efektivitātes fizikālo novērtējumu un citu mērījumu rezultātiem.\\
Rezultāti liecina, ka optimālie parametri ir LG panelis D virziena 40 grādu leņķī. Pētījuma datu analīzes rīks ir pieejams \url{https://github.com/chararchter/solR}.

\keywords{Saules enerģijas paneļi, atjaunojamo energoresursu enerģija}