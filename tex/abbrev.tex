\noindent 
\textbf{Klimats}\\
TSI - Kopējais saules apstarojums, $\textrm{Wm}^{-2}$\\
SSI - Saules spektrālais apstarojums, $\textrm{Wm}^{-2}\textrm{nm}^{-1}$\\
% $G_{sc}$ - Solārā konstante, $\textrm{Wm}^{-2}$\\
%Photovoltaic power potential
TIM - Kopējā apstarojuma novērotājs\\
GHI – Globālais horizontālais apstarojums,  $\textrm{kWh/m}^2$\\ %Global horizontal irradiation
DHI – Difūzais horizontālais apstarojums,  $\textrm{kWh/m}^2$\\ 
DNI – Tiešais normālais apstarojums, $\textrm{kWh/m}^2$\\ %Direct normal irradiation
CMF - Mākoņu modifikācijas reizinātājs\\
AU - astronomiskā vienība\\  %astronomical unit
\textbf{Saules kustības leņķi}\\
$\theta$ - staru krišanas leņķis uz saules paneli\\
$\delta$ - Saules deklinācija -- leņķis starp virzieniem uz Sauli un debess ekvatoru solārajā pusdienlaikā.\\
 $\phi$  - ģeogrāfiskais platums; pozitīvs Z virzienā.\\
$\beta$  - paneļa slīpums -- leņķis starp Saules paneļa virsmu un horizontāli.\\
$\gamma$ - paneļa azimuts -- leņķis starp virsmas normāles projekciju uz horizontālo  plakni un D virzienu.\\
$\omega$ - solārais stundu leņķis -- leņķis starp Saules stara virziena projekciju uz horizontālo plakni un D virzienu; negatīvs no rīta.\\
\textbf{Saules paneļi}\\
PV - fotoelektriskais elements\\ %fotogalvanisks?
Si - silīcijs\\
P - jauda, 	W\\
Pmax - maksimālā jauda, W\\
PVOUT – Saules fotoelementa potenciālā jauda, $\textrm{kWh/kWp}$\\ 
%Photovoltaic power potential
E$_{norm}$, $\textrm{kWh/m}^2$ - enerģija normēta uz saules paneļa laukuma vienību.\\
\textbf{Debespuses}\\
A - austrumi\\
R - rietumi\\
D - dienvidi\\