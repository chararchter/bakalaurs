\noindent TSI - Kopējais saules apstarojums, $\textrm{Wm}^{-2}$\\
SSI - Saules spektrālais apstarojums, $\textrm{Wm}^{-2}\textrm{nm}^{-1}$\\
% $G_{sc}$ - Solārā konstante, $\textrm{Wm}^{-2}$\\
PV - fotoelektriskais elements\\ %fotogalvanisks?
Si - silīcijs
PVOUT – Saules fotoelementa potenciālā jauda, $\textrm{kWh/kWp}$\\ %Photovoltaic power potential
GHI – Globālais horizontālais apstarojums,  $\textrm{kWh/m}^2$\\ %Global horizontal irradiation
DNI – Tiešais normālais apstarojums, $\textrm{kWh/m}^2$\\ %Direct normal irradiation
AU - astronomiskā vienība\\  %astronomical unit
P - jauda, 	W\\
Pmax - maksimālā jauda, W\\
% Fizikālo lielumu apzīmējumi + solārās kustības leņķi
\textbf{Debespuses}
A - austrumi\\
R - rietumi\\
D - dienvidi\\
\textbf{Saules kustības leņķi}\\
$\theta$ - staru krišanas leņķis uz saules paneli.\\
$\delta$ - Saules deklinācija --- leņķis starp virzieniem uz Sauli un uz debess ekvatoru solārajā pusdienlaikā, pozitīvs Z virzienā.\\
 $\phi$  - ģeogrāfiskais platums, pozitīvs Z virzienā.\\
$\beta$  - paneļa slīpums --- leņķis starp Saules paneļa virsmu un horizontāli.\\
$\gamma$ - paneļa azimuts --- leņķis starp virsmas normāles projekciju uz horizontālu  plakni un D virzienu, negatīvs A virzienā.\\
$\omega$ - solārais stundu leņķis --- leņķis starp Saules stara virziena projekciju uz horizontālu plakni un D virzienu (kas mainās Zemes rotācijas ap  savu asi dēļ), negatīvs no rīta.