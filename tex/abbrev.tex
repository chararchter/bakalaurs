\noindent TSI - Kopējais saules apstarojums, $\textrm{Wm}^{-2}$\\
SSI - Saules spektrālais apstarojums, $\textrm{Wm}^{-2}\textrm{nm}^{-1}$\\
% $G_{sc}$ - Solārā konstante, $\textrm{Wm}^{-2}$\\
PV - Saules fotoelements\\ %fotogalvanisks?
PVOUT – Saules fotoelementa potenciālā jauda, $\textrm{kWh/kWp}$\\ %Photovoltaic power potential
GHI – Globālais horizontālais apstarojums,  $\textrm{kWh/m}^2$\\ %Global horizontal irradiation
DIF – Difūzais horizontālais apstarojums  $\textrm{kWh/m}^2$\\ %Diffuse horizontal irradiation
GTI – Globālais apstarojums virsmai optimālā slīpumā, $\textrm{kWh/m}^2$\\ %Global irradiation for optimally tilted surface
OPTA – Optimālais slīpums enerģijas ieguves maksimizēšanai gada griezumā, °\\ %Optimum tilt to maximize yearly yield
DNI – Direct normal irradiation, $\textrm{kWh/m}^2$\\ %Direct normal irradiation


% radiometri\\
% ACRIM - Aktīvā Dobuma Radiometrs Apstarojuma Monitoringam (Active Cavity Radiometer for Irradiance Monitoring)
% ACRIMSAT - Aktīvā Dobuma Radiometera Apstarojuma Monitoringa Satelīts (Active Cavity Radiometer Irradiance Monitor Satellite) 
% VIRGO - Saules apstarojuma variabilitāte un gravitācijas oscilācijas (Variability of solar Irradiance and Gravity Oscillations)
% organizācijas
% NOAA - Okeanoloģijas un atmosfēras valsts pārvalde (National Oceanic and Atmospheric Administration)
% NASA - Nacionālā aeronautikas un kosmosa administrācija  (National Aeronautics and Space Administration)