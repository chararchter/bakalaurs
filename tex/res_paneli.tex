\section{Efektivitātes atkarība no parametriem}
\subsection{Saules paneļa tipa} \label{subsection:tipi}

Pēc tabulām \ref{tab:JA} un \ref{tab:LG}, kā arī grafikiem, kas integrēti diskusijā par citu parametru ietekmi uz efektivitāti (skat. \ref{subsection:gads}. nod.), redzams, ka LG tipa paneļi konsekventi ir efektīvāki par JA tipu. Tas ir saistīts gan ar kristāla veidu -- kā pierādīts \ref{section:tipi}. nod.,  monokristāliska Si paneļi ir ražīgāki par polikristāliem --, gan paneļu maksimālajām jaudām -- LG tā ir lielāka nekā JA.

\begin{table}[h]
    \caption{JA tipa paneļu saražotā enerģija uz kvadrātmetru\\ salīdzināta ar piranometra izmērīto enerģiju}
    \begin{center}
    %%%%%%%%%%%%%%%%%%%%%%%%%%%%%%%%%%%%%%%%%%%%%%%%%%%%%%%%%%%%%%%%%%%%%%
%%                                                                  %%
%%  This is a LaTeX2e table fragment exported from Gnumeric.        %%
%%                                                                  %%
%%%%%%%%%%%%%%%%%%%%%%%%%%%%%%%%%%%%%%%%%%%%%%%%%%%%%%%%%%%%%%%%%%%%%%
\begin{tabular}{ | c | r r r r r  r | } \hline
E, $\textrm{kWhm}^{-2}$	&A.13	&R.13	&D.13	&D.40	&D.90	&piranometrs\\ \hline
jan		&0.35	&0.23	&0.72	&2.46	&2.80		&12.14\\
feb		&3.20	&2.74	&4.27	&6.45	&5.99		&25.14\\
mar		&8.22	&7.40	&9.47	&11.94	&8.72		&61.76\\
apr		&19.89	&19.23	&23.27	&25.43	&18.25		&141.41\\ \hline
$E_{sum}$, $\textrm{kWhm}^{-2}$
		&31.7	&29.6	&37.7	&46.3	&35.8 	&240.5\\ \hline
\end{tabular}
    \end{center} \label{tab:JA}
\end{table}
\begin{table}[h]
    \caption{LG tipa paneļu saražotā enerģija uz kvadrātmetru\\ salīdzināta ar piranometra izmērīto enerģiju}
    \begin{center}
    %%%%%%%%%%%%%%%%%%%%%%%%%%%%%%%%%%%%%%%%%%%%%%%%%%%%%%%%%%%%%%%%%%%%%%
%%                                                                  %%
%%  This is a LaTeX2e table fragment exported from Gnumeric.        %%
%%                                                                  %%
%%%%%%%%%%%%%%%%%%%%%%%%%%%%%%%%%%%%%%%%%%%%%%%%%%%%%%%%%%%%%%%%%%%%%%
\begin{tabular}{ | c | r r r r r  r | } \hline
E, $\textrm{kWhm}^{-2}$	&A.13	&R.13	&D.13	&D.40	&D.90 	&piranometrs\\ \hline
jan	&0.5	&0.35	&0.95	&2.82	&3.22		&12.14	\\
feb	&4.43	&3.63	&5.05	&8.25	&7.41		&25.14	\\
mar	&10.92	&9.27	&11.27	&15.69	&11.34		&61.76	\\
apr	&27.63	&25.46	&29.14	&34.21	&23.18		&141.41	\\ \hline
$E_{sum}$, $\textrm{kWhm}^{-2}$
	&43.48	&38.7	&46.41	&60.98	&45.14		&240.46	\\ \hline
\end{tabular}
    \end{center} \label{tab:LG}
\end{table}

% %%%%%%%%%
\subsection{Saules paneļa leņķa}\label{subsection:degree}
Apkopojot četru mēnešu datus un abus paneļu tipus, visražīgākais leņķis ir 40\textdegree.
Kopumā var secināt, ka paneļi 90\textdegree ~leņķī ir saražo vairāk enerģijas ziemas mēnešos un 13\textdegree ~leņķī -- vasaras mēnešos, tātad apstrinās teorijā prognozētais (skat.\ref{section:kustiba}.nod).
% Tas sakrīt ar [ielikt grafiku ar LV karti un optimālo leņķi bet es neatceros no kurienes paņēmu to].
\begin{figure}[h]
    \centering
    \includegraphics[width=\linewidth]{figures/results/all_degType.pdf}
    \caption{D virzienā vērsto saules paneļu saražotā enerģija atkarībā no leņķa un saules paneļu tipa} \label{fig:lg_ja_deg}
\end{figure}


\subsection{Saules paneļa virziena}\label{subsection:dir}
Visražīgākais virziens ir D, tad A, tad R. To paskaidro \ref{subsection:month_day}. nod. redzamie paneļu saražotās enerģijas dienas sadalījumi.
% no idea why tho
\begin{figure}[h]
    \centering
    \includegraphics[width=\linewidth]{figures/results/all_dirType.pdf}
    \caption{13 grādu leņķī vērsto saules paneļu atkarība no virziena un saules paneļu tipa}
    \label{fig:lg_ja_dir}
\end{figure}

\subsection{Saules paneļa leņķa un virziena kombinācijas}
\label{subsection:month_day}

Saules paneļu saņemtā apstarojuma dienas sadalījuma tendenci drīkst salīdzināt ar attiecīgo paneļu saražotās enerģijas tendenci, jo pēdējā ir atkarīga no paneļa virsmas saņemtā Saules apstarojuma, kas savukārt ir funkcija no $cos(\theta)$, tātad proporcionāla tam. Salīdzinot \ref{fig:cos-theta}.(c) ar 
\ref{fig:toldU}, tiek secināts, ka eksperimentālie rezultāti sakrīt ar teoriju.

\begin{figure}[h]
    \centering
    \includegraphics[width=\linewidth]{figures/sol_day/apr_LG_13.pdf}
    \includegraphics[width=\linewidth]{figures/sol_day/apr_LG_D.pdf}
    \caption{Saules paneļu saražotās enerģijas dienas sadalījuma līkne vidējotiem 27-30. aprīļa datiem} \label{fig:toldU}
\end{figure}

% \begin{figure}[h]
%     \centering
%     \includegraphics[width=\linewidth]{figures/sol_day/feb_A13JA.pdf}
%     \includegraphics[width=\linewidth]{figures/sol_day/feb_R13JA.pdf}
%     \caption{A un R virzienu saules paneļu 5 minūtēs vidējotu Wh dienas sadalījumi februārī}
%     \label{fig:feb_ar}
% \end{figure}

% Kā redzams \ref{fig:feb_ar}, \ref{fig:mar_ar}, \ref{fig:apr_ar}. att., eksperimentāli noteiktais dienas sadalījums sakrīt ar teorētiski prognozēto \ref{fig:cos-theta}.att.
% \begin{figure}[h]
%     \centering
%     \includegraphics[width=\linewidth]{figures/sol_day/mar_A13JA.pdf}
%     \includegraphics[width=\linewidth]{figures/sol_day/mar_R13JA.pdf}
%     \caption{A un R virzienu saules paneļu 5 minūtēs vidējotu Wh dienas sadalījumi martā}
%     \label{fig:mar_ar}
% \end{figure}

% \begin{figure}[h]
%     \centering
%     \includegraphics[width=\linewidth]{figures/sol_day/apr_A13JA.pdf}
%     \includegraphics[width=\linewidth]{figures/sol_day/apr_R13JA.pdf}
%     \caption{A un R virzienu saules paneļu 5 minūtēs vidējotu Wh dienas sadalījumi aprīlī}
%     \label{fig:apr_ar}
% \end{figure}

% % \begin{figure}[h]
% %     \centering
% %     \includegraphics[width=\linewidth]{figures/sol_month/feb_Dir_d.pdf}
% %     \includegraphics[width=\linewidth]{figures/sol_month/feb_Deg_d.pdf}
% %     \caption{Saules paneļu saražotā enerģija atkarībā no virziena un leņķa februārī}
% %     \label{fig:feb_degDir}
% % \end{figure}

% % \begin{figure}[h]
% %     \centering
% %     \includegraphics[width=\linewidth]{figures/sol_month/mar_Dir_d.pdf}
% %     \includegraphics[width=\linewidth]{figures/sol_month/mar_Deg_d.pdf}
% %     \caption{Saules paneļu saražotā enerģija atkarībā no virziena un leņķa martā}
% %     \label{fig:mar_degDir}
% % \end{figure}

% % \begin{figure}[h]
% %     \centering
% %     \includegraphics[width=\linewidth]{figures/sol_month/apr_Dir_d.pdf}
% %     \includegraphics[width=\linewidth]{figures/sol_month/apr_Deg_d.pdf}
% %     \caption{Saules paneļu saražotā enerģija atkarībā no virziena un leņķa aprīlī}
% %     \label{fig:apr_degDir}
% % \end{figure}

\subsection{Gada mēneša} \label{subsection:gads}
Pēc \ref{fig:jan_sum}, \ref{fig:feb_sun}, \ref{fig:apr_sum}, \ref{fig:apr_sum}. att., tiek izdarīti secinājumi par saules paneļu ražīguma atkarību no gada mēneša. Janvārī visražīgākais panelis ir D.90, otrs ražīgākais - D.40, kas atbilst janvārim raksturīgajai Saules diennakts kustībai -- zemu pie horzionta. \ref{fig:feb_sun}. att. redzams, ka februārī D.40 kļūst ražīgāks nekā D.90, tāpat redzams, ka mazāku leņķu paneļi -- R.13 un A.13 -- ir palielinājuši ražīgumu. Šī tendence turpinās arī marta un aprīļa mēnešos.

\begin{figure}[h]
    \centering
    \includegraphics[width=\linewidth]{figures/sol_month/jan_m_m2.pdf}
    \caption{Saules paneļu saražotā enerģija janvārī}
    \label{fig:jan_sum}
\end{figure}

\begin{figure}[h]
    \centering
    \includegraphics[width=\linewidth]{figures/sol_month/feb_m_m2.pdf}
    \caption{Saules paneļu saražotā enerģija februārī}
    \label{fig:feb_sum}
\end{figure}

% \begin{figure}[h]
%     \centering
%     \includegraphics[width=\linewidth]{figures/sol_month/mar_m_m2.pdf}
%     \caption{Saules paneļu saražotā enerģija martā}
%     \label{fig:mar_sum}
% \end{figure}

\begin{figure}[h]
    \centering
    \includegraphics[width=\linewidth]{figures/sol_month/apr_m_m2.pdf}
    \caption{Saules paneļu saražotā enerģija aprīlī}
    \label{fig:apr_sum}
\end{figure}


% \subsection{Efektivitāte}\label{subsection:effectivity}

% Izmērītā paneļu efektivitāte atšķiras no ražotāju tehniskajā dokumentācijā dotā. Atšķirības tiek skaidrotas ar efektivitātes mērīšanas veidu -- standarta testi tiek veikti pie konstanta izstarojuma (1000 W/m$^2$ un 800 W/m$^2$), tomēr reālā poligona apstākļos saņemtais starojums ir mainīgs.
% % Iespējams tāpēc, ka visu paneļu saules apstarojuma references punktu izvēlējos eksperimentālā poligona meteostacijas datus un nepiereizināju tiem panelim atbilstošo leņķi, jo saules apstarojums mainās no leņķa.
% % you better do it fast 
% \begin{figure}[h]
%     \centering
%     \includegraphics[width=\linewidth]{figures/results/ja_m2.pdf}
%     \caption{JA tipa paneļu saražotais mēnesī salīdzinājumā ar eksperimentālā poligona meteostacijas stacijas saules apstarojuma datiem (sarkanā krāsā)}
%     \label{fig:ja}
% \end{figure}
% \begin{table}[h]
%     \caption{JA tipa paneļu efektivitāte procentos}
%     \begin{center}
%     %%%%%%%%%%%%%%%%%%%%%%%%%%%%%%%%%%%%%%%%%%%%%%%%%%%%%%%%%%%%%%%%%%%%%%
%%                                                                  %%
%%  This is a LaTeX2e table fragment exported from Gnumeric.        %%
%%                                                                  %%
%%%%%%%%%%%%%%%%%%%%%%%%%%%%%%%%%%%%%%%%%%%%%%%%%%%%%%%%%%%%%%%%%%%%%%
\begin{tabular}{ | c | c c c c c | } \hline
E, $\%$	&A.13	&R.13	&D.13	&D.40	&D.90\\ \hline
jan	&2.86	&1.87	&5.90	&20.27	&23.08\\
feb	&12.72	&10.91	&16.97	&25.65	&23.82\\
mar	&13.31	&11.98	&15.34	&19.33	&14.11\\
apr	&14.06	&13.60	&16.45	&17.98	&12.91\\ \hline
vid	&10.74	&9.59	&13.67	&20.81	&18.48\\ \hline
\end{tabular}
%     \end{center}
%     \label{tab:JA_eff}
% \end{table}

% \begin{figure}[h]
%     \centering
%     \includegraphics[width=\linewidth]{figures/results/lg_m2.pdf}
%     \caption{LG tipa paneļu saražotais mēnesī salīdzinājumā ar eksperimentālā poligona meteostacijas stacijas saules apstarojuma datiem (sarkanā krāsā)}
%     \label{fig:lg}
% \end{figure}
% \begin{table}[h]
%     \caption{LG tipa paneļu efektivitāte procentos}
%     \begin{center}
%     %%%%%%%%%%%%%%%%%%%%%%%%%%%%%%%%%%%%%%%%%%%%%%%%%%%%%%%%%%%%%%%%%%%%%%
%%                                                                  %%
%%  This is a LaTeX2e table fragment exported from Gnumeric.        %%
%%                                                                  %%
%%%%%%%%%%%%%%%%%%%%%%%%%%%%%%%%%%%%%%%%%%%%%%%%%%%%%%%%%%%%%%%%%%%%%%
\begin{tabular}{ | c | c c c c c | }\hline
E, $\%$	&A.13	&R.13	&D.13	&D.40	&D.90\\ \hline
jan		&4.1	&2.9	&7.8	&23.2	&26.5\\
feb		&17.6	&14.4	&20.1	&32.8	&29.5\\
mar		&17.7	&15.0	&18.2	&25.4	&18.4\\
apr		&19.5	&18.0	&20.6	&24.2	&16.4\\ \hline
vid		&14.7	&12.6	&16.7	&26.4	&22.7\\ \hline
\end{tabular}
%     \end{center}
%     \label{tab:LG_eff}
% \end{table}