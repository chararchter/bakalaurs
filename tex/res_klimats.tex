\section{Saules apstarojums}
% \begin{figure}[h]
%     \centering
%     \includegraphics[width=\linewidth]{figures/meteo/statsYears.pdf}
%     \caption{Solārā apstarojuma laika integrāļa atšķirības gada gaitā. Eksperimentālā poligona meteostacijas stacijas dati 2013 -- 2019 periodā.}
%     \label{fig:metYears}
% \end{figure}

% izlabo visur lu meteoroloģiju uz poligonu
% pārsaukt grafikā Em^-2 uz E_{norm} (normētais enerģijas blīvums)

Tieši ilgtermiņa saules apstarojuma monitorings saules paneļu uzstādīšanas vietā ir svarīgs paneļu efektivitātes prognozēšanas rīks, jo ļauj izslēgt anomālu saulaina laika periodu svaru, kādi, piemēram, ir 2014. un 2019. gada aprīļi (skat. \ref{fig:metYears_mean}. att.). Attēlos \ref{fig:met_Irrad} un \ref{fig:met_Irrad_mean} redzams dienas saules apstarojumai izmaiņas novēroto četru mēnešu laikā, kas izskaidro turpmāk minētās atšķirības paneļu ražīgumā atkarībā no leņķa.

\begin{figure}[h]
    \centering
    \includegraphics[width=\linewidth]{figures/meteo/meanYears.pdf}
    \caption{Solārā apstarojuma laika integrāļa atšķirības gada gaitā un to vidējās vērtības. Eksperimentālā poligona meteostacijas stacijas dati 2013 -- 2019 periodā.}
    \label{fig:metYears_mean}
\end{figure}
\begin{figure}[h]
    \centering
    \includegraphics[width=\linewidth]{figures/meteo/sun19.pdf}
    \caption{Solārā apstarojuma izmaiņas dienas gaitā. Eksperimentālā poligona meteostacijas stacijas dati 2019-01-01 -- 2019-04-31 periodā.}
    \label{fig:met_Irrad}
\end{figure}
\begin{figure}[h]
    \centering
    \includegraphics[width=\linewidth]{figures/meteo/mean19.pdf}
    \caption{Mēnesī vidējotas solārā apstarojuma izmaiņas dienas gaitā. Eksperimentālā poligona meteostacijas stacijas dati 2019-01-01 -- 2019-04-31 periodā.}
    \label{fig:met_Irrad_mean}
\end{figure}