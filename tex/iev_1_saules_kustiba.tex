% !TeX spellcheck = lv_LV
% intensity
% spectral distribution
% solar geometry
% saules stāvoklis debesīs
% un virziens, kurā stara starojums krīt uz dažādu virzienu un ēnojuma virsmām

\subsection{Saules stāvoklis debesīs}
Apzīmēsim ar $\theta$ staru krišanas leņķi uz Saules paneli, pieņemot, ka Saules panelis ir plakans un nekustīgs. Tad, pie nemainīgās starojuma intensitātes, paneļa saņemtā enerģija būs proporcionāla $\cos{\theta}$ (ja $\theta<90^\circ$) vai būs vienāda ar 0 (ja $\theta>90^\circ$, t.i. Saules stari krīt uz paneļa apakšējo virsmu). Saules diennakts kustība, gadalaiku cikls, ka arī Saules paneļa novietojums ir ievēroti izteiksmē, kas saskaņā ar [Solar\_Engineering\_of\_Thermal\_Processes.pdf] ļauj aprēķināt $\cos{\theta}$:
\begin{equation}
\label{eq:theta}
\begin{aligned}
	\cos{\theta} = {} & \sin{\delta} \sin{\phi} \cos{\beta} - \sin{\delta} \cos{\phi} \sin{\beta} \cos{\gamma} +                           \\
	                  & \cos{\delta} \cos{\phi} \cos{\beta} \cos{\omega} + \cos{\delta} \sin{\phi} \sin{\beta} \cos{\gamma} \cos{\omega} + \\
	                  & \cos{\delta} \sin{\beta} \sin{\gamma} \sin{\omega},
\end{aligned}
\end{equation}
kur lietoti leņķi, kas definēti \ref{tab:theta} tabulā. Saules deklināciju solārajā pusdienlaikā var aprēķināt pēc formulas
\begin{equation}
\label{eq:delta}
    \delta = 23 \sin \left( 360 \cdot \frac{284+n}{365} \right),
\end{equation}
kur $n$ ir dienas kārtas numurs gadā.

Ar vienādojumu \ref{eq:theta} un \ref{eq:delta} palīdzību ir iespējams aprēķināt $\cos{\theta}$ atkarības no laika, kas ir pirmais tuvinājums Saules apstarojuma izmaiņām dienas laikā.
% Parametru vērtības katram no darbā lietotajiem Saules paneļiem ir apkopotas \ref{tab:param}. tabulā.
% Lietojot šos parametrus, tika aprēķinātas $\cos{\theta}$ atkarības no laika diviem datumiem: 1. janvārim un 31. aprīlim, sk. \ref{fig:cos-theta}. attēlā. Var redzēt, ka modelis paredz būtiskākas vienkāršas sakarības. Piemēram, austrumu virzienā vērsts panelis "ieslēdzas" agrāk par rietumu virzienā vērstu paneli. Otrkārt, janvārī no D paneļiem visefektīvākais ir vertikālais, jo Saule atrodas zemu, savukārt aprīlī visefektīvākais ir 40$^\circ$ leņķis.

\begin{table}[h!]
	\caption{Leņķu, kas lietoti \ref{eq:theta} vienādojumā, definīcijas.}
	\begin{center}
		\begin{tabular}{|c|c|l|}
			\hline
			         &         apgabals         & definīcija                                                                 \\ \hline\hline
			$\theta$ &  $(0^\circ;180^\circ)$   & staru krišanas leņķis uz Saules paneli                                     \\ \hline
			$\delta$ &  $(-23^\circ;23^\circ)$  & Saules deklinācija --- leņķis starp virzieniem uz Sauli un uz debess       \\
			         &                          & ekvatoru solārajā pusdienlaikā, pozitīvs Z virzienā                        \\ \hline
			 $\phi$  &  $(-90^\circ;90^\circ)$  & ģeogrāfiskais platums, pozitīvs Z virzienā                                 \\ \hline
			$\beta$  &  $(0^\circ;180^\circ)$   & paneļa slīpums --- leņķis starp Saules paneļa virsmu un horizontāli        \\ \hline
			$\gamma$ & $(-180^\circ;180^\circ)$ & paneļa azimuts --- leņķis starp virsmas normāles projekciju uz horizontālu \\
			         &                          & plakni un D virzienu, negatīvs A virzienā                                  \\ \hline
			$\omega$ & $(-180^\circ;180^\circ)$ & solārais stundu leņķis --- leņķis starp Saules stara virziena projekciju   \\
			         &                          & uz horizontālu plakni un D virzienu (kas mainās Zemes rotācijas ap         \\
			         &                          & savu asi dēļ), negatīvs no rīta                                            \\ \hline
		\end{tabular}
	\end{center}
	\label{tab:theta}
\end{table}

% \begin{table}[h!]
% 	\caption{Darbā lietotajiem Saules paneļiem atbilstošās leņķisko parametru vērtības, grādos.}
% 	\begin{center}
% 		\begin{tabular}{|r|c|c|c|c|c|}
% 			\hline
% 			         & R.13 & A.13 &   D.13   & D.40 & D.90 \\ \hline\hline
% 			paneļa slīpums $\beta$  & \multicolumn{3}{c|}{13} &  40  &  90  \\ \hline
% 			paneļa azimuts $\gamma$ &  90  & -90  & \multicolumn{3}{c|}{0}  \\ \hline
% 			ģeogrāfiskais platums $\phi$  &        \multicolumn{5}{c|}{57}        \\ \hline
% 		\end{tabular}
% 	\end{center}
% 	\label{tab:param}
% \end{table}

% \begin{figure}[h]
% 	\centering
% 	\includegraphics[width=\linewidth]{figures/misc/cos-theta-jan.pdf}
% 	\includegraphics[width=\linewidth]{figures/misc/cos-theta-apr.pdf}
% 	\caption{Diennakts laikā paredzētas $\cos(\theta)$ vērtības darbā lietotajiem Saules paneļiem, aprēķinātas pēc \ref{eq:theta} izteiksmes 1. janvārim (augšā) un 30. aprīlim (apakšā).}
% 	\label{fig:cos-theta}
% \end{figure}