% TEORIJAS KOPSAVILKUMS
Vispirms tiek aplūkota Saules emitētā starojuma daba un ģeometriskie apsvērumi - virziens, no kura staru kūlis sasniedz virsmu, leņķis uz virsmas un laika gaitā saņemtais starojuma daudzums. Tiek apskatīta atmosfēras un mākoņu ietekme uz virsmas saņemto saules starojumu, un tās praktiskā nozīme, apstrādājot pieejamos Saules starojuma datus, lai aprakstītu radiācijas gadījumus uz virsmas dažādās orientācijās.

% atjaunojamā enerģija
% kāpēc ir svarīgi
% 	politika (direktīva)
% 	klimata pārmaiņas
% atjaunojamās enerģijas veidi (vējš, saule, utml)
% mērķis ir salīdzināt solāro paneļu saražoto enerģiju atkarībā no modeļa, telpiskās orientācijas, leņķa un kā mainās pa gadalaikiem tieši latvijā
% citos klimatiskajos apstākļos līdzīgas analīzes ir veiktas ko var iegūt dažādās klimatiskajās zonās
% pierakstīt par pv paneļu darbību un efektivitāti
% aprakstīt ko tu darīji, lai pēc iespējas automatizētu datu apstrādes sistēmu
% kādus datus uzkrāj kādā attēlojumā
% datu menedžmentu aprakstīt
% paskatīties kā nosaka solāro paneļu efektivitātes standartu
% sarēķināt attiecības starp paneļu saražoto (normēt uz lielāko paneli)
% ielikt grafikus no februāra un aprīļa
% pierakstīt salīdzinājumu
% aprakstīt gļukus, kur dienā nav nekā saražots bet blakus ir. tā gadās.
% nomaini februāra sol_month grafikos virsrakstu uz februāri
% nomaini $Em^-2$ uz $E_norm$